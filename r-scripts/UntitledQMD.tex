% Options for packages loaded elsewhere
\PassOptionsToPackage{unicode}{hyperref}
\PassOptionsToPackage{hyphens}{url}
\PassOptionsToPackage{dvipsnames,svgnames,x11names}{xcolor}
%
\documentclass[
  letterpaper,
  DIV=11,
  numbers=noendperiod]{scrartcl}

\usepackage{amsmath,amssymb}
\usepackage{iftex}
\ifPDFTeX
  \usepackage[T1]{fontenc}
  \usepackage[utf8]{inputenc}
  \usepackage{textcomp} % provide euro and other symbols
\else % if luatex or xetex
  \usepackage{unicode-math}
  \defaultfontfeatures{Scale=MatchLowercase}
  \defaultfontfeatures[\rmfamily]{Ligatures=TeX,Scale=1}
\fi
\usepackage{lmodern}
\ifPDFTeX\else  
    % xetex/luatex font selection
\fi
% Use upquote if available, for straight quotes in verbatim environments
\IfFileExists{upquote.sty}{\usepackage{upquote}}{}
\IfFileExists{microtype.sty}{% use microtype if available
  \usepackage[]{microtype}
  \UseMicrotypeSet[protrusion]{basicmath} % disable protrusion for tt fonts
}{}
\makeatletter
\@ifundefined{KOMAClassName}{% if non-KOMA class
  \IfFileExists{parskip.sty}{%
    \usepackage{parskip}
  }{% else
    \setlength{\parindent}{0pt}
    \setlength{\parskip}{6pt plus 2pt minus 1pt}}
}{% if KOMA class
  \KOMAoptions{parskip=half}}
\makeatother
\usepackage{xcolor}
\setlength{\emergencystretch}{3em} % prevent overfull lines
\setcounter{secnumdepth}{-\maxdimen} % remove section numbering
% Make \paragraph and \subparagraph free-standing
\ifx\paragraph\undefined\else
  \let\oldparagraph\paragraph
  \renewcommand{\paragraph}[1]{\oldparagraph{#1}\mbox{}}
\fi
\ifx\subparagraph\undefined\else
  \let\oldsubparagraph\subparagraph
  \renewcommand{\subparagraph}[1]{\oldsubparagraph{#1}\mbox{}}
\fi

\usepackage{color}
\usepackage{fancyvrb}
\newcommand{\VerbBar}{|}
\newcommand{\VERB}{\Verb[commandchars=\\\{\}]}
\DefineVerbatimEnvironment{Highlighting}{Verbatim}{commandchars=\\\{\}}
% Add ',fontsize=\small' for more characters per line
\usepackage{framed}
\definecolor{shadecolor}{RGB}{241,243,245}
\newenvironment{Shaded}{\begin{snugshade}}{\end{snugshade}}
\newcommand{\AlertTok}[1]{\textcolor[rgb]{0.68,0.00,0.00}{#1}}
\newcommand{\AnnotationTok}[1]{\textcolor[rgb]{0.37,0.37,0.37}{#1}}
\newcommand{\AttributeTok}[1]{\textcolor[rgb]{0.40,0.45,0.13}{#1}}
\newcommand{\BaseNTok}[1]{\textcolor[rgb]{0.68,0.00,0.00}{#1}}
\newcommand{\BuiltInTok}[1]{\textcolor[rgb]{0.00,0.23,0.31}{#1}}
\newcommand{\CharTok}[1]{\textcolor[rgb]{0.13,0.47,0.30}{#1}}
\newcommand{\CommentTok}[1]{\textcolor[rgb]{0.37,0.37,0.37}{#1}}
\newcommand{\CommentVarTok}[1]{\textcolor[rgb]{0.37,0.37,0.37}{\textit{#1}}}
\newcommand{\ConstantTok}[1]{\textcolor[rgb]{0.56,0.35,0.01}{#1}}
\newcommand{\ControlFlowTok}[1]{\textcolor[rgb]{0.00,0.23,0.31}{#1}}
\newcommand{\DataTypeTok}[1]{\textcolor[rgb]{0.68,0.00,0.00}{#1}}
\newcommand{\DecValTok}[1]{\textcolor[rgb]{0.68,0.00,0.00}{#1}}
\newcommand{\DocumentationTok}[1]{\textcolor[rgb]{0.37,0.37,0.37}{\textit{#1}}}
\newcommand{\ErrorTok}[1]{\textcolor[rgb]{0.68,0.00,0.00}{#1}}
\newcommand{\ExtensionTok}[1]{\textcolor[rgb]{0.00,0.23,0.31}{#1}}
\newcommand{\FloatTok}[1]{\textcolor[rgb]{0.68,0.00,0.00}{#1}}
\newcommand{\FunctionTok}[1]{\textcolor[rgb]{0.28,0.35,0.67}{#1}}
\newcommand{\ImportTok}[1]{\textcolor[rgb]{0.00,0.46,0.62}{#1}}
\newcommand{\InformationTok}[1]{\textcolor[rgb]{0.37,0.37,0.37}{#1}}
\newcommand{\KeywordTok}[1]{\textcolor[rgb]{0.00,0.23,0.31}{#1}}
\newcommand{\NormalTok}[1]{\textcolor[rgb]{0.00,0.23,0.31}{#1}}
\newcommand{\OperatorTok}[1]{\textcolor[rgb]{0.37,0.37,0.37}{#1}}
\newcommand{\OtherTok}[1]{\textcolor[rgb]{0.00,0.23,0.31}{#1}}
\newcommand{\PreprocessorTok}[1]{\textcolor[rgb]{0.68,0.00,0.00}{#1}}
\newcommand{\RegionMarkerTok}[1]{\textcolor[rgb]{0.00,0.23,0.31}{#1}}
\newcommand{\SpecialCharTok}[1]{\textcolor[rgb]{0.37,0.37,0.37}{#1}}
\newcommand{\SpecialStringTok}[1]{\textcolor[rgb]{0.13,0.47,0.30}{#1}}
\newcommand{\StringTok}[1]{\textcolor[rgb]{0.13,0.47,0.30}{#1}}
\newcommand{\VariableTok}[1]{\textcolor[rgb]{0.07,0.07,0.07}{#1}}
\newcommand{\VerbatimStringTok}[1]{\textcolor[rgb]{0.13,0.47,0.30}{#1}}
\newcommand{\WarningTok}[1]{\textcolor[rgb]{0.37,0.37,0.37}{\textit{#1}}}

\providecommand{\tightlist}{%
  \setlength{\itemsep}{0pt}\setlength{\parskip}{0pt}}\usepackage{longtable,booktabs,array}
\usepackage{calc} % for calculating minipage widths
% Correct order of tables after \paragraph or \subparagraph
\usepackage{etoolbox}
\makeatletter
\patchcmd\longtable{\par}{\if@noskipsec\mbox{}\fi\par}{}{}
\makeatother
% Allow footnotes in longtable head/foot
\IfFileExists{footnotehyper.sty}{\usepackage{footnotehyper}}{\usepackage{footnote}}
\makesavenoteenv{longtable}
\usepackage{graphicx}
\makeatletter
\def\maxwidth{\ifdim\Gin@nat@width>\linewidth\linewidth\else\Gin@nat@width\fi}
\def\maxheight{\ifdim\Gin@nat@height>\textheight\textheight\else\Gin@nat@height\fi}
\makeatother
% Scale images if necessary, so that they will not overflow the page
% margins by default, and it is still possible to overwrite the defaults
% using explicit options in \includegraphics[width, height, ...]{}
\setkeys{Gin}{width=\maxwidth,height=\maxheight,keepaspectratio}
% Set default figure placement to htbp
\makeatletter
\def\fps@figure{htbp}
\makeatother

\KOMAoption{captions}{tableheading}
\makeatletter
\@ifpackageloaded{caption}{}{\usepackage{caption}}
\AtBeginDocument{%
\ifdefined\contentsname
  \renewcommand*\contentsname{Table of contents}
\else
  \newcommand\contentsname{Table of contents}
\fi
\ifdefined\listfigurename
  \renewcommand*\listfigurename{List of Figures}
\else
  \newcommand\listfigurename{List of Figures}
\fi
\ifdefined\listtablename
  \renewcommand*\listtablename{List of Tables}
\else
  \newcommand\listtablename{List of Tables}
\fi
\ifdefined\figurename
  \renewcommand*\figurename{Figure}
\else
  \newcommand\figurename{Figure}
\fi
\ifdefined\tablename
  \renewcommand*\tablename{Table}
\else
  \newcommand\tablename{Table}
\fi
}
\@ifpackageloaded{float}{}{\usepackage{float}}
\floatstyle{ruled}
\@ifundefined{c@chapter}{\newfloat{codelisting}{h}{lop}}{\newfloat{codelisting}{h}{lop}[chapter]}
\floatname{codelisting}{Listing}
\newcommand*\listoflistings{\listof{codelisting}{List of Listings}}
\makeatother
\makeatletter
\makeatother
\makeatletter
\@ifpackageloaded{caption}{}{\usepackage{caption}}
\@ifpackageloaded{subcaption}{}{\usepackage{subcaption}}
\makeatother
\ifLuaTeX
  \usepackage{selnolig}  % disable illegal ligatures
\fi
\usepackage{bookmark}

\IfFileExists{xurl.sty}{\usepackage{xurl}}{} % add URL line breaks if available
\urlstyle{same} % disable monospaced font for URLs
\hypersetup{
  pdftitle={Summary of dplyr functionality},
  pdfauthor={Daan Kroeze - ChatGPT},
  colorlinks=true,
  linkcolor={blue},
  filecolor={Maroon},
  citecolor={Blue},
  urlcolor={Blue},
  pdfcreator={LaTeX via pandoc}}

\title{Summary of dplyr functionality}
\author{Daan Kroeze - ChatGPT}
\date{2024-09-11}

\begin{document}
\maketitle

The \texttt{dplyr} package in R is one of the most widely used packages
for data manipulation and transformation. It provides a set of
easy-to-use functions designed to work efficiently with data frames (or
tibbles). The primary goal of \texttt{dplyr} is to simplify common data
manipulation tasks using a consistent syntax.

Here's a summary of key \texttt{dplyr} functions and their
functionality:

\begin{center}\rule{0.5\linewidth}{0.5pt}\end{center}

\subsubsection{\texorpdfstring{1. \textbf{Basic Data Manipulation
Functions:}}{1. Basic Data Manipulation Functions:}}\label{basic-data-manipulation-functions}

\paragraph{\texorpdfstring{\textbf{1.1
\texttt{select()}}}{1.1 select()}}\label{select}

\begin{itemize}
\item
  \textbf{Purpose}: Select specific columns from a data frame.
\item
  \textbf{Example}:

\begin{Shaded}
\begin{Highlighting}[]
\NormalTok{df }\SpecialCharTok{\%\textgreater{}\%} \FunctionTok{select}\NormalTok{(column1, column2)}
\end{Highlighting}
\end{Shaded}
\end{itemize}

\paragraph{\texorpdfstring{\textbf{1.2
\texttt{filter()}}}{1.2 filter()}}\label{filter}

\begin{itemize}
\item
  \textbf{Purpose}: Filter rows based on specific conditions.
\item
  \textbf{Example}:

\begin{Shaded}
\begin{Highlighting}[]
\NormalTok{df }\SpecialCharTok{\%\textgreater{}\%} \FunctionTok{filter}\NormalTok{(column1 }\SpecialCharTok{\textgreater{}} \DecValTok{10}\NormalTok{, column2 }\SpecialCharTok{==} \StringTok{"value"}\NormalTok{)}
\end{Highlighting}
\end{Shaded}
\end{itemize}

\paragraph{\texorpdfstring{\textbf{1.3
\texttt{mutate()}}}{1.3 mutate()}}\label{mutate}

\begin{itemize}
\item
  \textbf{Purpose}: Add new columns or modify existing ones.
\item
  \textbf{Example}:

\begin{Shaded}
\begin{Highlighting}[]
\NormalTok{df }\SpecialCharTok{\%\textgreater{}\%} \FunctionTok{mutate}\NormalTok{(}\AttributeTok{new\_column =}\NormalTok{ column1 }\SpecialCharTok{*} \DecValTok{2}\NormalTok{)}
\end{Highlighting}
\end{Shaded}
\end{itemize}

\paragraph{\texorpdfstring{\textbf{1.4
\texttt{arrange()}}}{1.4 arrange()}}\label{arrange}

\begin{itemize}
\item
  \textbf{Purpose}: Sort rows by one or more columns.
\item
  \textbf{Example}:

\begin{Shaded}
\begin{Highlighting}[]
\NormalTok{df }\SpecialCharTok{\%\textgreater{}\%} \FunctionTok{arrange}\NormalTok{(column1)  }\CommentTok{\# Ascending}
\NormalTok{df }\SpecialCharTok{\%\textgreater{}\%} \FunctionTok{arrange}\NormalTok{(}\FunctionTok{desc}\NormalTok{(column1))  }\CommentTok{\# Descending}
\end{Highlighting}
\end{Shaded}
\end{itemize}

\paragraph{\texorpdfstring{\textbf{1.5 \texttt{summarize()} (or
\texttt{summarise()})}}{1.5 summarize() (or summarise())}}\label{summarize-or-summarise}

\begin{itemize}
\item
  \textbf{Purpose}: Summarize data by applying aggregation functions
  such as \texttt{sum()}, \texttt{mean()}, \texttt{min()},
  \texttt{max()}, etc.
\item
  \textbf{Example}:

\begin{Shaded}
\begin{Highlighting}[]
\NormalTok{df }\SpecialCharTok{\%\textgreater{}\%} \FunctionTok{summarize}\NormalTok{(}\AttributeTok{total =} \FunctionTok{sum}\NormalTok{(column1))}
\end{Highlighting}
\end{Shaded}
\end{itemize}

\paragraph{\texorpdfstring{\textbf{1.6
\texttt{group\_by()}}}{1.6 group\_by()}}\label{group_by}

\begin{itemize}
\item
  \textbf{Purpose}: Group data by one or more columns to perform grouped
  operations (e.g., summarize by group).
\item
  \textbf{Example}:

\begin{Shaded}
\begin{Highlighting}[]
\NormalTok{df }\SpecialCharTok{\%\textgreater{}\%} \FunctionTok{group\_by}\NormalTok{(group\_column) }\SpecialCharTok{\%\textgreater{}\%} \FunctionTok{summarize}\NormalTok{(}\AttributeTok{total =} \FunctionTok{sum}\NormalTok{(column1))}
\end{Highlighting}
\end{Shaded}
\end{itemize}

\paragraph{\texorpdfstring{\textbf{1.7
\texttt{rename()}}}{1.7 rename()}}\label{rename}

\begin{itemize}
\item
  \textbf{Purpose}: Rename columns in a data frame.
\item
  \textbf{Example}:

\begin{Shaded}
\begin{Highlighting}[]
\NormalTok{df }\SpecialCharTok{\%\textgreater{}\%} \FunctionTok{rename}\NormalTok{(}\AttributeTok{new\_name =}\NormalTok{ old\_name)}
\end{Highlighting}
\end{Shaded}
\end{itemize}

\paragraph{\texorpdfstring{\textbf{1.8
\texttt{distinct()}}}{1.8 distinct()}}\label{distinct}

\begin{itemize}
\item
  \textbf{Purpose}: Select distinct (unique) rows from a data frame.
\item
  \textbf{Example}:

\begin{Shaded}
\begin{Highlighting}[]
\NormalTok{df }\SpecialCharTok{\%\textgreater{}\%} \FunctionTok{distinct}\NormalTok{(column1)}
\end{Highlighting}
\end{Shaded}
\end{itemize}

\begin{center}\rule{0.5\linewidth}{0.5pt}\end{center}

\subsubsection{\texorpdfstring{2. \textbf{Advanced
Functions:}}{2. Advanced Functions:}}\label{advanced-functions}

\paragraph{\texorpdfstring{\textbf{2.1
\texttt{slice()}}}{2.1 slice()}}\label{slice}

\begin{itemize}
\item
  \textbf{Purpose}: Select rows by position (e.g., the first row, last
  row, or a range of rows).
\item
  \textbf{Example}:

\begin{Shaded}
\begin{Highlighting}[]
\NormalTok{df }\SpecialCharTok{\%\textgreater{}\%} \FunctionTok{slice}\NormalTok{(}\DecValTok{1}\SpecialCharTok{:}\DecValTok{5}\NormalTok{)  }\CommentTok{\# First 5 rows}
\end{Highlighting}
\end{Shaded}
\end{itemize}

\paragraph{\texorpdfstring{\textbf{2.2
\texttt{slice\_max()}}}{2.2 slice\_max()}}\label{slice_max}

\begin{itemize}
\item
  \textbf{Purpose}: Select rows with the maximum value in a column.
\item
  \textbf{Example}:

\begin{Shaded}
\begin{Highlighting}[]
\NormalTok{df }\SpecialCharTok{\%\textgreater{}\%} \FunctionTok{slice\_max}\NormalTok{(column1, }\AttributeTok{n =} \DecValTok{3}\NormalTok{)  }\CommentTok{\# Top 3 rows with the highest values}
\end{Highlighting}
\end{Shaded}
\end{itemize}

\paragraph{\texorpdfstring{\textbf{2.3
\texttt{slice\_min()}}}{2.3 slice\_min()}}\label{slice_min}

\begin{itemize}
\item
  \textbf{Purpose}: Select rows with the minimum value in a column.
\item
  \textbf{Example}:

\begin{Shaded}
\begin{Highlighting}[]
\NormalTok{df }\SpecialCharTok{\%\textgreater{}\%} \FunctionTok{slice\_min}\NormalTok{(column1, }\AttributeTok{n =} \DecValTok{2}\NormalTok{)  }\CommentTok{\# Top 2 rows with the lowest values}
\end{Highlighting}
\end{Shaded}
\end{itemize}

\paragraph{\texorpdfstring{\textbf{2.4
\texttt{sample\_n()}}}{2.4 sample\_n()}}\label{sample_n}

\begin{itemize}
\item
  \textbf{Purpose}: Randomly select a specific number of rows.
\item
  \textbf{Example}:

\begin{Shaded}
\begin{Highlighting}[]
\NormalTok{df }\SpecialCharTok{\%\textgreater{}\%} \FunctionTok{sample\_n}\NormalTok{(}\DecValTok{5}\NormalTok{)  }\CommentTok{\# Randomly select 5 rows}
\end{Highlighting}
\end{Shaded}
\end{itemize}

\paragraph{\texorpdfstring{\textbf{2.5
\texttt{sample\_frac()}}}{2.5 sample\_frac()}}\label{sample_frac}

\begin{itemize}
\item
  \textbf{Purpose}: Randomly sample a fraction of the rows.
\item
  \textbf{Example}:

\begin{Shaded}
\begin{Highlighting}[]
\NormalTok{df }\SpecialCharTok{\%\textgreater{}\%} \FunctionTok{sample\_frac}\NormalTok{(}\FloatTok{0.1}\NormalTok{)  }\CommentTok{\# Randomly select 10\% of the rows}
\end{Highlighting}
\end{Shaded}
\end{itemize}

\paragraph{\texorpdfstring{\textbf{2.6
\texttt{pull()}}}{2.6 pull()}}\label{pull}

\begin{itemize}
\item
  \textbf{Purpose}: Extract a single column as a vector.
\item
  \textbf{Example}:

\begin{Shaded}
\begin{Highlighting}[]
\NormalTok{df }\SpecialCharTok{\%\textgreater{}\%} \FunctionTok{pull}\NormalTok{(column\_name)}
\end{Highlighting}
\end{Shaded}
\end{itemize}

\begin{center}\rule{0.5\linewidth}{0.5pt}\end{center}

\subsubsection{\texorpdfstring{3. \textbf{Join Functions (for Combining
Data):}}{3. Join Functions (for Combining Data):}}\label{join-functions-for-combining-data}

\texttt{dplyr} provides several join functions similar to SQL for
combining data from multiple data frames:

\paragraph{\texorpdfstring{\textbf{3.1
\texttt{left\_join()}}}{3.1 left\_join()}}\label{left_join}

\begin{itemize}
\item
  \textbf{Purpose}: Returns all rows from the left table and matching
  rows from the right table.
\item
  \textbf{Example}:

\begin{Shaded}
\begin{Highlighting}[]
\NormalTok{df1 }\SpecialCharTok{\%\textgreater{}\%} \FunctionTok{left\_join}\NormalTok{(df2, }\AttributeTok{by =} \StringTok{"key\_column"}\NormalTok{)}
\end{Highlighting}
\end{Shaded}
\end{itemize}

\paragraph{\texorpdfstring{\textbf{3.2
\texttt{right\_join()}}}{3.2 right\_join()}}\label{right_join}

\begin{itemize}
\item
  \textbf{Purpose}: Returns all rows from the right table and matching
  rows from the left table.
\item
  \textbf{Example}:

\begin{Shaded}
\begin{Highlighting}[]
\NormalTok{df1 }\SpecialCharTok{\%\textgreater{}\%} \FunctionTok{right\_join}\NormalTok{(df2, }\AttributeTok{by =} \StringTok{"key\_column"}\NormalTok{)}
\end{Highlighting}
\end{Shaded}
\end{itemize}

\paragraph{\texorpdfstring{\textbf{3.3
\texttt{inner\_join()}}}{3.3 inner\_join()}}\label{inner_join}

\begin{itemize}
\item
  \textbf{Purpose}: Returns only the matching rows between both tables.
\item
  \textbf{Example}:

\begin{Shaded}
\begin{Highlighting}[]
\NormalTok{df1 }\SpecialCharTok{\%\textgreater{}\%} \FunctionTok{inner\_join}\NormalTok{(df2, }\AttributeTok{by =} \StringTok{"key\_column"}\NormalTok{)}
\end{Highlighting}
\end{Shaded}
\end{itemize}

\paragraph{\texorpdfstring{\textbf{3.4
\texttt{full\_join()}}}{3.4 full\_join()}}\label{full_join}

\begin{itemize}
\item
  \textbf{Purpose}: Returns all rows when there is a match in either
  left or right table.
\item
  \textbf{Example}:

\begin{Shaded}
\begin{Highlighting}[]
\NormalTok{df1 }\SpecialCharTok{\%\textgreater{}\%} \FunctionTok{full\_join}\NormalTok{(df2, }\AttributeTok{by =} \StringTok{"key\_column"}\NormalTok{)}
\end{Highlighting}
\end{Shaded}
\end{itemize}

\paragraph{\texorpdfstring{\textbf{3.5
\texttt{anti\_join()}}}{3.5 anti\_join()}}\label{anti_join}

\begin{itemize}
\item
  \textbf{Purpose}: Returns rows from the left table that do not have a
  match in the right table.
\item
  \textbf{Example}:

\begin{Shaded}
\begin{Highlighting}[]
\NormalTok{df1 }\SpecialCharTok{\%\textgreater{}\%} \FunctionTok{anti\_join}\NormalTok{(df2, }\AttributeTok{by =} \StringTok{"key\_column"}\NormalTok{)}
\end{Highlighting}
\end{Shaded}
\end{itemize}

\begin{center}\rule{0.5\linewidth}{0.5pt}\end{center}

\subsubsection{\texorpdfstring{4. \textbf{Piping
(\texttt{\%\textgreater{}\%})}}{4. Piping (\%\textgreater\%)}}\label{piping}

The \texttt{\%\textgreater{}\%} operator, also known as the pipe
operator, is used to chain together multiple \texttt{dplyr} functions in
a readable and efficient way. Instead of nesting functions inside one
another, you pass the result of one function to the next function in a
sequence.

\paragraph{\texorpdfstring{\textbf{Example:}}{Example:}}\label{example}

\begin{Shaded}
\begin{Highlighting}[]
\NormalTok{df }\SpecialCharTok{\%\textgreater{}\%}
  \FunctionTok{filter}\NormalTok{(column1 }\SpecialCharTok{\textgreater{}} \DecValTok{10}\NormalTok{) }\SpecialCharTok{\%\textgreater{}\%}
  \FunctionTok{select}\NormalTok{(column1, column2) }\SpecialCharTok{\%\textgreater{}\%}
  \FunctionTok{arrange}\NormalTok{(}\FunctionTok{desc}\NormalTok{(column1))}
\end{Highlighting}
\end{Shaded}

\begin{center}\rule{0.5\linewidth}{0.5pt}\end{center}

\subsubsection{\texorpdfstring{5. \textbf{Common Aggregation Functions
Used in
\texttt{dplyr}:}}{5. Common Aggregation Functions Used in dplyr:}}\label{common-aggregation-functions-used-in-dplyr}

\begin{itemize}
\tightlist
\item
  \texttt{sum()}: Sum of values.
\item
  \texttt{mean()}: Mean (average) of values.
\item
  \texttt{median()}: Median of values.
\item
  \texttt{min()}, \texttt{max()}: Minimum or maximum value.
\item
  \texttt{n()}: Number of observations in a group.
\item
  \texttt{n\_distinct()}: Count of unique values.
\item
  \texttt{sd()}: Standard deviation of values.
\end{itemize}

\begin{center}\rule{0.5\linewidth}{0.5pt}\end{center}

\subsubsection{Summary:}\label{summary}

The \texttt{dplyr} package is highly flexible and efficient for data
manipulation tasks such as: - Selecting and filtering rows. - Creating
and modifying columns. - Summarizing and aggregating data. - Sorting and
rearranging rows. - Joining multiple data frames.

Its combination with the pipe operator (\texttt{\%\textgreater{}\%})
enables you to write clean, readable code for complex data workflows.
Let me know if you'd like more details on any specific functions or
operations!



\end{document}
